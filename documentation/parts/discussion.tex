\section{Discussion} 
The project kicked-off with five team members in the first meeting. After the mid-term presentation, in which the attendance of each team member was mandatory, one member had left the team, reducing the team size to four members.
During the development the importance of intense code reviews and cooperate thinking became even more clearer than expected and already known, due to different levels of experiences of reviews in the team. A large amount of time has been invested in refactoring and restructuring to changed requirements by new features, which made reviews an essential part in the development with the side effect of learning and understanding the programming style of other developers.
Experiments and training with new (and often unknown) technologies and frameworks required a larger time investment in the first half of the development than expected, reducing the target amount of features at the end. A specific example is the usage of Docker and Docker-Compose which might have resulted in a too ambiguous overhead for a project of such a size, as multiple team members had problems running the docker-based project in the beginning. A traditional server would probably have been less problematical. Nonetheless, the development of a Docker-based multi-component server had been a valuable experience in software deployment and in regards of developing in a full-stack manner.