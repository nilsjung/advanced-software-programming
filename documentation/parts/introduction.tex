\newpage
\section{Introduction}
As part of the mandatory course “Advanced Software Programming” the task of a accompanying JavaScript-based project has been given with the goal to reach abilities of full stack-development. Plain JavaScript is a well-developed area with best-practices for most applications. Nevertheless the team decided to experiment with state-of-art technologies to familiarize itself with modern and well-used frameworks and tools such as React-Redux, Bootstrap and Docker. Nowadays web developers often are confronted with those broadly used technologies which gave the team additional attraction as research objectives:
\begin{itemize}
	\item Styling and structuring frameworks
	\item Frontend frameworks (like Vue.js, ReactJS, AngularJS)
	\item Backend frameworks for NodeJS (like Express, Koa)
	\item Toolchain
\end{itemize}
Beside those team individual requirements of experimenting with modern frameworks given requirements were:
\begin{itemize}
	\item Git as version control service
	\item User sign-up and authentication
	\item Database access
	\item Websockets
\end{itemize}
The development management was based on modern iterative approaches and used support tools and services for cooperative development such as Github and the git-flow workflow, Kanban boards with tickets (user stories, bug tickets) and code reviews.

\subsection{Initial Phase}
During the first meeting, after the selection of team members, the discussion about the concrete application began. The development of a module based chat application has been selected for multiple reasons: encapsulated development of features as modules on a basic chat, which can be extended independently during the development. After the decision the first User-Stories were described and documented to fix the vision of the application and to serve as the first project specific requirements:
\begin{itemize}
	\item Basic chat functionality
	\item Translation functionality within a message
	\item Usage of emojis
	\item Markdown support
	\item Users can join chat rooms
	\item modifiable user settings
	\item online status
\end{itemize}

For the initial commit each team member was given the opportunity to familiarize himself with technologies while building a rather simple chat application. These prototypes served as candidates for the initial version of the whole team from which one got selected which already contained primitive React-Redux functionalities. From this point forward the created requirements had been imported into the GitHub ticket system and each member was able to start developing features by the Git workflow.
